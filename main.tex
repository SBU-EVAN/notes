\documentclass{article}
\usepackage{geometry}
 \geometry{
 a4paper,
 total={170mm,257mm},
 left=20mm,
 top=20mm,
 }
\usepackage[utf8]{inputenc}
\usepackage{hyperref}
\usepackage{braket}
\usepackage{amsmath}
\usepackage{amssymb}
\usepackage{amsthm}
\usepackage{mathtools}
\usepackage{amsfonts}
\usepackage{physics}
\usepackage{tikz-cd}
\usepackage{musicography}
\usepackage{enumerate}
\usepackage{xfrac}
\usepackage[parfill]{parskip}
\usepackage{subcaption}
\usepackage{float}
\usepackage{pgfplots}
\pgfplotsset{compat=1.12}
\usepgfplotslibrary{fillbetween}

\newcommand{\cm}{\mathbb{C}}
\newcommand{\re}{\mathbb{R}}
\newcommand{\dvg}{\nabla\cdot}
\newcommand{\crl}{\nabla\times}
\newcommand{\proj}[1]{\ensuremath{{\ket{#1}\bra{#1}}}}
\newcommand{\vect}{\text{Vect}}
\newcommand{\smooth}{C^{\infty}}
\newcommand{\dif}{\text{d}}

\makeatletter
\newcommand*\bigcdot{\mathpalette\bigcdot@{.8}}
\newcommand*\bigcdot@[2]{\mathbin{\vcenter{\hbox{\scalebox{#2}{$\m@th#1\bullet$}}}}}
\makeatother
\newlength{\arrow}
\settowidth{\arrow}{\scriptsize$1000$}
\newcommand*{\myrightarrow}[1]{\xrightarrow{\mathmakebox[\arrow]{#1}}}
\newcommand*{\TakeFourierOrnament}[1]{{%
\fontencoding{U}\fontfamily{futs}\selectfont\char#1}}
\newcommand*{\danger}{\TakeFourierOrnament{66}}
\setlength{\jot}{1.5mm}

\theoremstyle{definition}
\newtheorem{thm}{Theorem}[section] % the main one
\newtheorem{lemma}[thm]{Lemma}
\newtheorem{joke}{Joke}
% other statement types

% for specifying a name
\theoremstyle{plain} % just in case the style had changed
\newcommand{\thistheoremname}{}
\newtheorem{genericthm}[thm]{\thistheoremname}
\newenvironment{namedthm}[1]
  {\renewcommand{\thistheoremname}{#1}%
   \begin{genericthm}}
  {\end{genericthm}}
  
\newtheorem{defn}{Definition}
\newtheorem{prop}{Proposition}
\newtheorem{rmk}{Remark}
\newtheorem{conj}{Conjecture}
%\newtheorem{lemma}{Lemma}
\newtheorem{cor}{Corollary}
\newtheorem{problem}{Problem}




\begin{document}

\begin{center}
	\huge{Research notes}
	
	\small{All here is public}
\end{center}

\section{Statistics}
\subsection{Tensions in Cosmology}
Define the posterior probability of parameters $\theta$ within model $\mathcal{M}$ given data $d$ as (Bayes' theorem)
\begin{equation}
    \mathcal{P}(\theta) := P(\theta|d,\mathcal{M}) = \frac{P(\theta|\mathcal{M})P(d|\theta,\mathcal{M})}{P(d|\mathcal{M})}
\end{equation}
The following notation will simplify this expression
\begin{equation*}
    \begin{split}
        \Pi(\theta) &= P(\theta|\mathcal{M}) \text{ , Prior probability} \\
        \mathcal{L}(\theta) &= P(d|\theta,\mathcal{M}) \text{ , the likelihood} \\
        \mathcal{E} &= P(d|\mathcal{M}) \text{ , the evidence}
    \end{split}
\end{equation*}
$$     \mathcal{P}(\theta) := P(\theta|d,\mathcal{M}) = \frac{\Pi(\theta)\mathcal{L}}{\mathcal{E}} $$
These are (maybe?) understandable as the prior probability is the parameters given the model, the likelihood as the data given some parameters of the model, and the evidence as the data give a model. From now on, the model $\mathcal{M}$ will be implied.

Consider now two data sets $d_1,d_2$. They have a joint likelihood $$\mathcal{L}(\theta) = P(d_1,d_2|\theta)$$
This \textbf{quantifies the likelihood of $d_1$ and $d_2$ coming from the same set of parameters of a given model.} Denote now $\mathcal{L}_1,\mathcal{P}_1$ as the marginalized likelihood and Posterior over $d_2$. (\textbf{Is there a procedure/algorithm to find the marginalized probability? Seems generally non-trivial}).

Now duplicate the parameter set to $\theta_1,\theta_2$ and perscibe a new joint likelihood $\mathcal{L}(\theta_1,\theta_2) = P(d_1,d_2|\theta_1,\theta_2)$.\textbf{ This is generally a choice, but how can one make a good choice?} The choice however is not unique.

As such, the constraints are imposed:
\begin{enumerate}
    \item $\mathcal{L}(\theta_1=\theta,\theta_2=\theta) = \mathcal{L}(\theta)$, or in plain english, the joint likelihood if $\theta_1=\theta=\theta_2$ needs to coincide with the likelihood of $\theta$
    \item $P(d_1|\theta_1,\theta_2)=P(d_1|\theta_1)$ once marginalized over $d_2$. Marginalizing the likelihood over one of the data sets removes the dependancy on the corresponding parameter sets.
\end{enumerate}
 \textbf{This ensures the datasets are conditionally independent,}
    $$ P(d_1,d_2|\theta) = P(d_1|\theta)P(d_2|\theta)$$
\textbf{Proof?}
Which means we can choose $$\mathcal{L}(\theta_1,\theta_2) = \mathcal{L}_1(\theta_1)\mathcal{L}_2(\theta_2)$$
If we further assume the prior distribution can be factorized, the joint posterior is then
$$ \mathcal{P}(\theta_1,\theta_2) = \mathcal{L}(\theta_1,\theta_2)\Pi(\theta_1)\Pi(\theta_2) $$
Defining $\Delta\theta = \theta_1-\theta_2$, the parameter difference posterior is given by
\[ \mathcal{P}(\Delta\theta) = \int_{V_\Pi} \mathcal{P}(\theta,\theta-\Delta\theta)d\theta\]

\section{Perturbation Theory}
\subsection{Scalar-Vector-Tensor Decomposition}
Our universe, as described above, is a real valued 4 dimensional space, $\mathbb{R}^4$. Suppose that we can separate the universe into a spacial part and a temporal part $\mathbb{R}^4 \mapsto \mathbb{R} \times S$ where $S$ is some $3$-manifold. Under such a decomposition, the metric decomposes as (with comoving time as the time coordinate)
\[ g = a^2(\tau)\left( g_{00}d\tau d\tau + g_{0i}d\tau dx^{i} + g_{ij} dx^{i}dx^{j} \right) \]
The three parts are as follows:
\begin{itemize}
    \item $g_{00}$ has degrees of freedom (DOF) of a scalar. This is the scalar portion of the decomposition.
    \item $g_{0i}$ has DOF of a vector. This is the (co)vector portion of the decomposition.
    \item $g_{ij}$ has DOF of a rank 2 tensor. This is the tensor portion of the decomposition.
\end{itemize}
The metric, however, is a special case of a symmetric rank 2 tensor. For antisymmetric tensors the only change is that rather than the product $dx^\mu dx^\nu$, we need the exterior product $dx^\mu \wedge dx^\nu$. This means the temporal component/scalar component is 0 for antisymmetric tensors. Since any rank 2 tensor can be decomposed to a symmetric and an antisymmetric part, these two decompositions are sufficient for decomposing any rank 2 tensor. 

We can take this even further. Note that a vector can be decomposed into a divergence part and a dual part
\[ v^i = g^{ij}\partial_j f + g^{ij} \underbrace{w_{jk}\star (dx^{i} \wedge dx^{j})}_{\equiv \hat{w}_j} \]
\[ \Rightarrow v^i = (\partial^j f + \hat{w}^j) \]
Also, any rank 2 tensor can be written as the sum of a trace and a traceless part.



\section{Computing}
\subsection{Markov Chain Monte Carlo (MCMC)} 
Interestingly, MCMC algorithms have heavy analogies with statistical mechanics which are useful to demonstrate the concept. To examine this, lets first define what a Markov Chain is.
\begin{defn}
A sequence $X_1,\hdots,X_n$ of random elements is a \textit{Markov Chain} if the conditional distribution $X_{n+1}$ depends only on $X_n$. The set in which $X_i$ take values is called the \textit{state space} of the chain.
\end{defn}


\subsection{The Metropolis-Hastings Algorithm}
Suppose we want to sample from a distribution $p(x)$. $p(x)$ can be high dimensional and is generally difficult to calculate (evidence is hard to compute since it requires integration over the entire parameter space). The goal is to use Markov Chains to sample from $p(x)$ without needing to compute the evidence. This will be represented as a path through state space until the chain reaches a stable point (stationary state).

We start with a proposal distribution $g(x_n)$. Sample from the proposal distribution to find the next state $x_{n+1}$ with probability $g(x_{n+1}|x_n)$. This transition from state $n$ to state $n+1$ must follow the \textit{detailed balance condition}
\[ p(x_n) g(x_{n+1}|x_n) A(x_n \rightarrow x_{n+1}) = p(x_{n+1}) g(x_n|x_{n+1}) A(x_{n+1} \rightarrow x_{n}) \]
where $A$ is an \textit{acceptance probability} which I will define more precisely later. Using Bayes' Theorem on $p(x)$, the evidence cancels out on each side, and thus the detailed balance condition can be simplified to only rely on the likelihood and prior of $p(x)$, which I will denote $\pi$ and $\mathcal{L}$.
\[ \pi(x_n)\mathcal{L}(x_n) g(x_{n+1}|x_n) A(x_n \rightarrow x_{n+1}) = \pi(x_{n+1}) \mathcal{L}(x_{n+1}) g(x_{n}|x_{n+1}) A(x_{n+1}\rightarrow x_{n}) \]
\[ \Rightarrow \frac{A(x_n \rightarrow x_{n+1})}{A(x_{n+1}\rightarrow x_{n})} = \frac{\pi(x_{n+1}) \mathcal{L}(x_{n+1}) g(x_{n}|x_{n+1})}{\pi(x_n)\mathcal{L}(x_n) g(x_{n+1}|x_n)} \equiv R_{n,n+1} \]
This allows us to define the acceptance probability as
\[ A(x_n \rightarrow x_{n+1}) = \min( 1, R_{n,n+1} ) \]
This probability is used to determine whether the chain moves to $x_{n+1}$ or stays at $x_n$. The chain converges when it reaches a stationary state.

There are a few properties that can be observed for this algorithm:
\begin{itemize}
    \item Having an asymmetrical proposal $g(x)$ can allow for faster convergence of the chain.
    \item The initial sampling may not accurately reflect samples for $p(x)$. This is regarded as the `burn-in' and is generally discarded from the samples.
    \item MCMC Sampling loses sampling power for multi-modal distributions. 
\end{itemize}

\subsection{Parameter Difference Distribution}
Define the posterior probability of parameters $\theta$ within model $\mathcal{M}$ given data $d$ as (Bayes' theorem)
\begin{equation}
    \mathcal{P}(\theta) := P(\theta|d,\mathcal{M}) = \frac{P(\theta|\mathcal{M})P(d|\theta,\mathcal{M})}{P(d|\mathcal{M})}
\end{equation}
The following notation will simplify this expression
\begin{equation*}
    \begin{split}
        \Pi(\theta) &= P(\theta|\mathcal{M}) \text{, Prior probability} \\
        \mathcal{L}(\theta) &= P(d|\theta,\mathcal{M}) \text{, the likelihood} \\
        \mathcal{E} &= P(d|\mathcal{M}) \text{, the evidence}
    \end{split}
\end{equation*}
\[ \mathcal{P}(\theta) := P(\theta|d,\mathcal{M}) = \frac{\Pi(\theta)\mathcal{L}}{\mathcal{E}} \]
These are (maybe?) understandable as the prior probability is the parameters given the model, the likelihood as the data given some parameters of the model, and the evidence as the data give a model. From now on, the model $\mathcal{M}$ will be implied.

Consider now two data sets $d_1,d_2$. They have a joint likelihood 
\[ \mathcal{L}(\theta) = P(d_1,d_2|\theta) \]
This \textbf{quantifies the likelihood of $d_1$ and $d_2$ coming from the same set of parameters of a given model.} Denote now $\mathcal{L}_1,\mathcal{P}_1$ as the marginalized likelihood and Posterior over $d_2$. (\textbf{Is there a procedure/algorithm to find the marginalized probability? Seems generally non-trivial}).

Now duplicate the parameter set to $\theta_1,\theta_2$ and perscibe a new joint likelihood $\mathcal{L}(\theta_1,\theta_2) = P(d_1,d_2|\theta_1,\theta_2)$. \textbf{This is generally a choice, but how can one make a good choice?} The choice however is not unique.

As such, the constraints are imposed:
\begin{enumerate}
    \item $\mathcal{L}(\theta_1=\theta,\theta_2=\theta) = \mathcal{L}(\theta)$, or in plain english, the joint likelihood if $\theta_1=\theta=\theta_2$ needs to coincide with the likelihood of $\theta$
    \item $P(d_1|\theta_1,\theta_2)=P(d_1|\theta_1)$ once marginalized over $d_2$. Marginalizing the likelihood over one of the data sets removes the dependancy on the corresponding parameter sets.
\end{enumerate}
 \textbf{This ensures the datasets are conditionally independent,}
\[ P(d_1,d_2|\theta) = P(d_1|\theta)P(d_2|\theta) \]
\textbf{Proof?}
Which means we can choose 
\[ \mathcal{L}(\theta_1,\theta_2) = \mathcal{L}_1(\theta_1)\mathcal{L}_2(\theta_2) \]
If we further assume the prior distribution can be factorized, the joint posterior is then
\[ \mathcal{P}(\theta_1,\theta_2) = \mathcal{L}(\theta_1,\theta_2)\Pi(\theta_1)\Pi(\theta_2) \]
Defining $\Delta\theta = \theta_1-\theta_2$, the parameter difference posterior is given by
\[ \mathcal{P}(\Delta\theta) = \int_{V_\Pi} \mathcal{P}(\theta,\theta-\Delta\theta)d\theta\]

\subsection{Quantifying Results}

Given some probability $P$ of a parameter shift, the following formula can give you the number of standard deviations if the probability shift comes from a gaussian distribution
\[ n_\sigma = \sqrt{2} \text{Erf}^{-1}(P) \]
I have a notebook using two unit gaussian priors separated by a distance $a$. This example can be computed analytically.
\begin{equation*}
    \begin{split}
	\mathcal{P}(\Delta \theta) &= \frac{1}{2\pi} \int\limits_{-\infty}^{\infty} e^{-\theta^2/2} e^{-{(\theta-\Delta\theta)}^2/2}  d\theta \\
				   &= \frac{1}{2\pi} \cdot \sqrt{\pi} e^{-{(\Delta\theta)}^2/4}\\
				   &= \frac{1}{\sqrt{4\pi}}e^{-{(\Delta\theta)}^2/4}\\
    \end{split}
\end{equation*}
The parameter difference posterior is a gaussian with standard deviation $\sqrt{2}$. The separation is fixed by $a$, hence the shift is $\mathcal{P}(a)$. Hence the shift probability is
\[ \Delta = \int\limits_{-a}^{a} e^{-{(\Delta\theta)}^2/4} d\Delta\theta \]
Lets use the example $a=2$. Then $n_\sigma = 2/\sqrt{2} = \sqrt{2}$. Using this we can work backwards to find $\Delta$ from a $z$-table to find $\Delta = 0.9207 - 0.0793 = 0.8414 $. 

\subsection{Normalizing Flows}
The method of normalizing flows (MAF) implemented here uses Masked Autoencoders (MADE) to construct the flow. Suppose we have an input to the flow $x_i$. The output of the map is $y_i= \mu(x_{1:i-1})+\sigma(x_{1:i-1})x_i$. The $\mu$ and $\sigma$ are found using neural networks which recieve masked inputs $x_{1:i-1}=(x_1,\ldots,x_{i-1},0,\ldots,0)$. Since the input only depends on the first $i-1$ inputs, the normalizing flow is \textit{autoregressive} and the Jacobian is triangular.

The implementation in tensorflow uses \textit{bijectors} which implements a local diffeomorphism between a manifold $M$ and a target manifold $N$ (which are our parameter spaces), i.e. $\phi:M\rightarrow N$ such that $\phi$ is differentiable and injective. In tensorflow it has three operations, Forward, Inverse, and log\_deg\_jacobian, which are exactly the three we want. By constructing a bijector for each masked input, the full normalizing map can be constructed.

\bigskip
\hrule
\textbf{\large References}

\url{https://www.colorado.edu/amath/sites/default/files/attached-files/2_28_2018.pdf}

\url{https://si.biostat.washington.edu/sites/default/files/modules/Geyer-Introduction\%20to\%20markov\%20chain\%20Monte\%20Carlo_0.pdf}

\url{https://towardsdatascience.com/monte-carlo-markov-chain-mcmc-explained-94e3a6c8de11}

\url{https://www.sheffield.ac.uk/polopoly_fs/1.60510!/file/MCMC.pdf}
\hrule

Test graph

  \begin{tikzpicture}

    \pgfplotsset{
      scale only axis,
    }

    \begin{axis}[
      xlabel=$x$,
      ylabel=$y$,
      samples=100,
      ]
      \addplot[][domain=-4.5:4]{0.2*x^3-x+1};
      \addplot[draw=red][domain=-4.5:4]{3*sin(deg(x))};
    \end{axis}

  \end{tikzpicture}


\section{Fields}
\subsection{Multifield Dark Energy}
So I don't keep having to look at this
\begin{defn}
Given a semi-riemannian manifold $M$ with metric $g$, the christoffel symbols are given by
$$ \nabla_a\partial_b = \Gamma^c_{ab}\partial_c $$
$$ \partial_ag_{bc} + \partial_bg_{ca} - \partial_cg_{ab} = 2g_{dc}\Gamma^d_{ab} $$
\end{defn}
Consider a metric of the form diag$(-1,a^2(t))$, so its determinant is $-a^6$.
The action is
$$ S=\int d^4x \sqrt{-g} \left[ \frac{1}{2}M_{p}^2 R -\frac{1}{2}\gamma_{ab}\partial_\mu\phi^a\partial^\mu\phi^b - V(\phi)+\mathcal{L}_m \right] $$
I only want to describe a homogeneous background, so the field is only a function of time. The action becomes
$$ S=\int d^4x \sqrt{-g} \left[ \frac{1}{2}M_{p}^2 R -\frac{1}{2}\gamma_{ab}\dot\phi^a\dot\phi^b - V(\phi)+\mathcal{L}_m \right] $$
Varying the field gives
$$ \delta S = \int d^4x \sqrt{-g}\left[ - \frac{1}{2}\partial_a(\gamma_{bc})\dot\phi^b\dot\phi^c - \frac{1}{2}\gamma_{ab}\frac{d}{dt}(\delta\phi^a)\dot\phi^b - \partial_aV\delta\phi^a \right] $$
$$ \delta S = -\frac{1}{2}(\delta\phi^a)(\sqrt{-g}\gamma_{ab}\dot\phi^b + \int d^4x \sqrt{-g}\left[ -\frac{1}{2}\gamma(\nabla_a\partial_b,\partial_c)\dot\phi^b,\dot\phi^c -\frac{1}{2}\gamma(\partial_b,\nabla_a\partial_c)\dot\phi^b\dot\phi^c + 3\frac{\dot a}{a}\gamma_{ab}\delta\phi^a\dot\phi^b + \gamma_{ab}\delta\phi^a\ddot\phi^b -V_a\delta\phi^a \right] $$
$$ \delta S = -\int d^4x \sqrt{-g}\left[ \frac{1}{2}\Gamma_{ab}^d\gamma_{dc}\dot\phi^b\dot\phi^c + \frac{1}{2}\Gamma_{ac}^d\gamma_{bd}\dot\phi^b\dot\phi^c + 3H\gamma_{ab}\dot\phi^b + \gamma_{ab}\ddot\phi^b + V_a  \right] $$
Multiply everything by $\gamma^{aa}$. Now lets do some reshuffling of the indices.
$$ \gamma^{aa}\Gamma^d_{ab}\gamma_{dc}\dot\phi^b\dot\phi^c $$
$$ a\leftrightarrow d$$
$$ \gamma^{ad}\Gamma^a_{db}\gamma_{ac}\dot\phi^b\dot\phi^c $$
$$ b\leftrightarrow d$$
$$ \gamma^{ad}\Gamma^a_{bd}\gamma_{ac}\dot\phi^b\dot\phi^c $$
$$ \gamma^{ab}\Gamma^a_{bd}\gamma_{ac}\dot\phi^d\dot\phi^c $$
$$ c\leftrightarrow d$$
$$ \gamma^{ab}\Gamma^a_{bc}\gamma_{ad}\dot\phi^d\dot\phi^c = \Gamma_{bc}^a \dot\phi^b\dot\phi^c $$
and
$$ \gamma^{aa} \Gamma^d_{ac}\gamma_{bd}\dot\phi^b\dot\phi^c $$
$$ a\leftrightarrow d$$
$$ \gamma^{ad} \Gamma^a_{dc}\gamma_{ba}\dot\phi^b\dot\phi^c $$
$$ b\leftrightarrow d$$
$$ \gamma^{ab} \Gamma^a_{bc}\gamma_{da}\dot\phi^d\dot\phi^c = \Gamma_{bc}^a\dot\phi^b\dot\phi^c $$
Thus the equation of motion is found,
$$ \ddot\phi^a + \Gamma_{bc}^a\dot\phi^b\dot\phi^c +3H\dot\phi^a + V^a = 0 $$
$$ D_t\dot\phi^a+3H\dot\phi^a + V^a = 0 $$

The riemann curvature is
$$ R^\alpha_{\beta\gamma\delta} = \partial_\beta\Gamma^\alpha_{\gamma\delta} - \partial_\gamma\Gamma^\alpha_{\beta\delta}+\Gamma^\sigma_{\gamma\delta}\Gamma^\alpha_{\beta\sigma} - \Gamma^\sigma_{\beta\delta}\Gamma^\alpha_{\gamma\sigma} $$
So the Ricci curvature tensor is
$$ R_{\beta\delta} = R^\alpha_{\beta\alpha\delta} = \partial_\beta\Gamma^\alpha_{\alpha\delta} - \partial_\alpha\Gamma^\alpha_{\beta\delta}+\Gamma^\sigma_{\alpha\delta}\Gamma^\alpha_{\beta\sigma} - \Gamma^\sigma_{\beta\delta}\Gamma^\alpha_{\alpha\sigma} $$
And the Ricci scalar curvature is
$$ R = R_{\beta}^{\beta} = g^{\beta\delta}R_{\beta\delta} =
g^{\beta\delta}\partial_\beta\Gamma^\alpha_{\alpha\delta} -
g^{\beta\delta}\partial_\alpha\Gamma^\alpha_{\beta\delta}
+g^{\beta\delta}\Gamma^\sigma_{\alpha\delta}\Gamma^\alpha_{\beta\sigma} -
g^{\beta\delta}\Gamma^\sigma_{\beta\delta}\Gamma^\alpha_{\alpha\sigma} $$
Since the fields only depend on time, we only need to consider the temporal component of the Ricci curvature tensor, so
$$ R_{tt} = \partial_t \Gamma^{\alpha}_{\alpha t} - \partial_\alpha\Gamma^{\alpha}_{tt}
+ \Gamma^{\sigma}_{\alpha t}\Gamma^{\alpha}_{t \sigma} - \Gamma^{\sigma}_{tt}\Gamma^{\alpha}_{\alpha\sigma} $$
From the definition of the christoffel symbols, and noting $ g = g(t)$ and $g$ is diagonal, we have $\partial_t g_{aa} = 2g_{ba}\Gamma^b_{ta}$, thus
the second term is necessarily 0 since $\partial_t g_{tt} = 0$. The first term is non-zero for $\alpha = 1,2,3$, in which it equals 
$$ \frac{1}{2}\partial_t (1/a^2 \partial_t a^2) = \partial_t(\dot a / a) = -3\ddot a /a + 3\dot a^2 /a^2 $$
The third term is non-zero for $\alpha = \sigma$, In which case we get 
$$ \frac{3}{4a^4}(\partial_t (a^2))^2 =  -3\dot a^2/a^2  $$
The last term is vanishes since the temporal component of the metric is constant. Thus we find
$$ R_{tt} = 3\frac{\ddot a}{a}$$
Now, if we consider spatial components of the Ricci curvature, we find that the first term now vanishes, the second term is equal to $ -\ddot a/a - \dot a^2/a^2$, the thrid term is 0, and the last term is $-\dot a^2/a^2$, so the spacial components are
$$ R_{xx} = -\ddot a  a - 2\dot a^2$$
Thus the Ricci curvature scalar is
$$ 6\ddot a/a + 6\dot a^2/a^2 $$
Hence the einstein equation gives
$$ -3H^2 =  $$
I need to write out the indices better if I want to get the signs correct.

\hrule
$$ \partial_ag_{bc} + \partial_bg_{ca} - \partial_cg_{ab} = 2g_{dc}\Gamma^d_{ab} $$

\hrule
Considering the way this acts on the coordinate vector field $\partial_\alpha$, 
$$ R\partial_\alpha =
g^{\beta\delta}\partial_\beta\Gamma^\alpha_{\alpha\delta}\partial_\alpha -
g^{\beta\delta}\partial_\alpha\Gamma^\alpha_{\beta\delta}\partial_\alpha
+g^{\beta\delta}\Gamma^\sigma_{\alpha\delta}\Gamma^\alpha_{\beta\sigma}\partial_\alpha -
g^{\beta\delta}\Gamma^\sigma_{\beta\delta}\Gamma^\alpha_{\alpha\sigma} \partial_\alpha $$
$$ R\partial_\alpha =
g^{\beta\delta}\partial_\beta \nabla_\alpha\partial_\delta -
g^{\beta\delta}\partial_\alpha \nabla_\beta\partial_\delta
+g^{\beta\delta}\Gamma^\sigma_{\alpha\delta} \nabla_\beta\partial_\sigma -
g^{\beta\delta}\Gamma^\sigma_{\beta\delta} \nabla_\alpha \partial_\sigma $$

\end{document}
